\documentclass{article}
\usepackage{graphicx} % Required for inserting images
\usepackage{tikz}
\usepackage{tcolorbox}


\title{Teoria dell'informazione\\Lezione I}
\author{Andrea Cosentino}
\date{31 December 2023}

\begin{document}

    \maketitle

    \section{Introduzione}

    Durante il corso del '900 ci sono state diverse persone che hanno sviluppato le fondamenta della disciplina. Tra queste ricordiamo quelli che vengono considerati i padri della disciplina:
    \begin{itemize}
        \item Claude Shannon(USA): primo in assoluto. Fa una definizione in media
        \item Kolmogorov(URSS): arriva dopo Shannon ma fa una definizione puntuale. Espande il suo lavoro.
        \item Chaitin e Solomonoff: arrivano allo stesso tempo di Kolmogorov ma non vengonono considerati perché Kolmogorov era, ed è, più importante a livello accademico.
    \end{itemize}
    \vspace{10px}
    Durante questo corso ci proponiamo di riuscire a spedire dei dati da una sorgente a una destinazione
    attraverso un canale che può essere affetto da rumore.
    \\ \\
    Obiettivi del corso:
    \begin{itemize}
        \item{Sfruttare al massimo il canale}
        \item{Gestire i bit persi nella trasmissione}
    \end{itemize}
    \vspace{10px}
    \section{Modellazione}
    (Sarebbe carino mettere un disegnino qui)\\
    Shannon modella l'ambiente come composto da 3 attori:
    \begin{itemize}
        \item \textbf{Sorgente}: La sorgente genera il messaggio, lo codifica e lo spedisce sul canale.
        \item \textbf{Canale}: Il canale è il tramite tra la sorgente e la destinazione. E' il "posto" in cui passa l'informazione. E' affetto da \textbf{rumore}.
        \item \textbf{Ricevente}: Riceve il messaggio codificato. E' suo compito riuscirlo a decodificare.
    \end{itemize}
    \vspace{10px}
    Vogliamo codificare messaggi sorgente \\ \\ (A) Massimizzando informazioni trasmesse A OGNI utilizzo del canale (problema di \textbf{Source coding}) \\ \\ (B) Minimizzando, simultaneamente al primo punto, il numero di errori di trasmissione dovuti al rumore (problema di \textbf{Channel coding}).
    \vspace{5px}
    Shannon cerca di risolvere questo problema usando l'approccio divide et impera. Approccio che non è detto sia quello giusto. Infatti la soluzione ottimale dei due sottoproblemi non è detto che, se messe assieme, diano la soluzione ottimale per il problema. Questo perché potremmo non sfruttare possibili vantaggi di un problema sull'altro.\\
    Questo non è il caso e infatti vale il seguente teorema.
    \vspace{5px}
    \begin{tcolorbox}
        \textbf{\textcolor{red}{TEOREMA} di codifica sorgente e canale}
        \vspace{5px}
        \begin{center}
            L'unione delle soluzioni di source coding e channel coding (quindi dei due sottoproblemi risolti come indipendenti) dà la soluzione ottima.
        \end{center}
    \end{tcolorbox}

    \vspace{10px}

    \noindent Come risolvo il source coding? Enunciamo, in maniera non formale per adesso, il primo teorema di Shannon.
    \vspace{5px}
    \begin{tcolorbox}
        \textbf{\textcolor{red}{TEOREMA} I teorema di Shannon}
        \vspace{5px}
        \begin{center}
            Si può comprimere, tramite un codice, un messaggio con perdite di informazioni piccole
        \end{center}
    \end{tcolorbox}
    \vspace{5px}
    \noindent Questo è dovuto al fatto che l'informazione non è uniformemente distribuita. Ci sono parti della codifica inutile.\\
    \vspace{5px}

    \noindent Esempio:\\
    Se codifico un cielo tutto azzurro, non ho bisogno di dire che ogni pixel è di colore azzurro, ma mi basta dire che una porzione di una foto è tutta azzurra.

\end{document}
