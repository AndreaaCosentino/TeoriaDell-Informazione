\documentclass{article}
\usepackage{graphicx} % Required for inserting images
\usepackage{tikz}
\usepackage{tcolorbox}
\usepackage{amsmath}
\usepackage{amssymb}
\usepackage{tkz-euclide}


\title{Teoria dell'informazione\\Lezione II}
\author{Andrea Cosentino}
\date{5 January 2024}

\begin{document}

    \maketitle

    \section{Limitazioni}

    La limitazione imposta la scorsa lezione, ovvero che il codice sia non singolare, non è sufficiente. Infatti, senza nessun'altra limitazione non possiamo decodificare un in modo univoco.  \\

    \noindent Esempio: La sorgente produde due simboli: A e B. \\
    Se A viene codificato in 0, e B in 00, il codice è non singolare, ma se ricevo 00, non so se la stringa inviata era B o AA. \\


    \noindent Introduciamo un'altra limitazione, questa volta sulla \textbf{estensione di un codice}.  L'estensione del codice $c\,:\, \mathbb{X}\,\rightarrow \, \mathbb{D}^+$ è definita come $$\mathbb{C}\, : \, \mathbb{X}^+ \, \rightarrow \mathbb{D}^+$$

    Imponiamo che il codice sia \textbf{univocamente decodificabile}. Si dice che il codice è univocamente decodifcabile se e solo se la sua estensione è non singolare. Questa proprietà implica che
    $$\forall \, y \, \epsilon \, \mathbb{D}^+ \,\,\text{c'è al più un  unico messaggio}
    x\,\epsilon\,\mathbb{X}^+ \,: \, \mathbb{C}(x) = y$$

    \noindent Per determinare se un codice è univocamente decodificabile si usa l'algoritmo di \textbf{Sardinas-Patterson}. L'algoritmo ha complessita O(mL), dove m è il numero delle parole di codice ($|\mathbb{X}|$) e L è la somma delle loro lunghezze ($\sum_{x\epsilon\mathbb{X}} l(x)$). \\

    \def\firstcircle{(0,0) circle (1.5cm)}
    \def\secondcircle{(5:1cm) circle (3cm)}
    \begin{tikzpicture}
        \begin{scope}[shift={(3cm,-5cm)}, fill opacity=1]
        \fill[white] \firstcircle;
        \fill[white] \secondcircle;
        \draw \firstcircle node[]  {\shortstack{Univocamente \\ decodificabile}};
        \draw \secondcircle node [right] {\hspace{2em}\shortstack{Non\\singolare}};
        \end{scope}
    \end{tikzpicture}
    \vspace{10px}

    \noindent In questo modo riusciamo a distinguere i simboli all'interno di una stringa in fase di decodifica. Il problema è che riusciamo solamente se abbiamo tutta la stringa. Questo non è sempre possibile (si pensi a situazioni in cui le informazioni arrivino in streaming senza terminare). \\ Dobbiamo aggiungere una limitazione che impedisca a una parola di codice di essere prefissa di un'altra. \\
    Introduciamo il concetto i \textbf{codice istantaneo}. Un codice si dice istantaneo se nessuna parola è prefissa di un'altra.

    \section{Codici istantanei}

    \vspace{5px}
    \begin{tcolorbox}
        \textbf{\textcolor{red}{FATTO}}
        \begin{center}
            Se c è istantaneo allora è anche univocamente decodificabile
        \end{center}
    \end{tcolorbox}

    \vspace{5px}

    \noindent \textbf{DIMOSTRAZIONE}
    \vspace{5px}

    \noindent Sia $c\,:\,\mathbb{X} \, \rightarrow \, \mathbb{D}^+$ e $\mathbb{C}$ la sua estensione. \\ \\
    Possiamo escludere il caso in cui c sia non singolare. Infatti, in questo caso, avrei 2 simboli che codificano nella stessa parola di codice. Due parole uguali sono l'una il prefisso dell'altra. \\ \\
    A questo punto resta da dimostrare che se $c$ non è univocamente decodificabile allora non è istantaneo. \\
    Assumiamo che c sia non univocamente decodificabile, quindi \begin{equation} \label{eq:one}
    \exists \, x \,,\,x^{'} \, \epsilon \, \mathbb{X}\, \text{con} \, x \neq x^{'} | \ \mathbb{C}(x) = \mathbb{C}(x^{'})
    \end{equation}
    $x$ e $x^{'}$ possono differire in 2 modi soltanto:
    \begin{itemize}
        \item Un messaggio è prefisso dell'altro
        \item C'è almeno uuna posizione in cui i 2 messaggi differiscono
    \end{itemize}
    Se $\mathbb{C}(x) = \mathbb{C}(x^{'})$ e $x$ è prefisso di $x^{'}$ (o viceversa) allora i restanti simboli di x dovrebbero \underline{per forza} essere mappati in parola vuota. \\ Ma questo non è possibile, per costruzione infatti sappiamo che un simbolo non può essere mappato in una parola vuota. Ma anche se cambiassimo la nostra definizione, e ammettessimo la parola vuota, questo ci porterebbe a dire che il codice è non istantaneo, perché la parola vuota è prefisso di tutte le parole.
    \\ \\
    L'unico caso possibile è quindi il secondo. \\ Siccome $x \neq x^{'}$, c'è una posizione i tale che $x_i \neq x_i^{'}$.
    \\ \\
    Allora fino alla posizione i, ovvreo per $j = 1,..,i-1$ si ha che
    $$\mathbb{C}(x_j) = \mathbb{C}(x_j^{i})$$
    Se però, $\mathbb{C}(x) = \mathbb{C}(x^{i})$, allora $c(x_i)$ è prefisso di $c(x_i^{'})$ (o viceversa). \\ \\

    \noindent Questo perché, se fino alla posizione j sono uguale, e invece a i sono diverso, condizione necessaria affinché valga  $\mathbb{C}(x) = \mathbb{C}(x^{i})$ è che da j in poi, qualunque cosa ci sia dopo, io la codifico uguale. Questo non può succedere se $c(x_i)$ non è prefisso di $c(x_i^{i})$ o viceversa.
    \\ \\
    Questo implica che il codice non è istantaneo.\\ Abbiamo dimostrato che $$\text{codici istantanei} \subset \text{codici univocamente decodificabili}$$


    \vspace{5px}
    \begin{tcolorbox}
        \textbf{\textcolor{red}{LEMMA} Disuguaglianza di Kraft}
        \begin{center}
            Dati $ \mathbb{X} = \{x_1,..,x_n\}, D > 1$ e m interi $l_1,..,l_m > 0$, esiste n codice istantaneo $c: \mathbb{X} \rightarrow \mathbb{D}^+$ tale che $l_c(x_i) = l_i$ per $i = 1,..,m \leftrightarrow $ $$\sum_{i = 1}^m D^{-l_i} \leq 1$$
        \end{center}
    \end{tcolorbox}

    \vspace{5px}

    \noindent Dimostro il lato $\rightarrow$. \\  \\
    Sia $l_{max} = \max_{i=1,...,m} l_c(x_i)$ \\
    Posso costruire un albero D-ario completo e di profondità $l_max$. Ogni nodo rappresnta una parola \textbf{possibile} di codice. Non è detto che sia utilizzata.



\end{document}
